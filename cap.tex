\documentclass{article}

% Preamble
\usepackage[dvipsnames]{xcolor}
\usepackage{geometry}
\usepackage{amsthm}
\usepackage{amsmath, amsfonts, amssymb}
\usepackage{txfonts,pxfonts}
\usepackage[many]{tcolorbox}
\usepackage{hyperref}
\hypersetup{
	colorlinks=true,
	linkcolor=blue,
	filecolor=red,
	urlcolor=cyan
}
\usepackage{}
\newtheorem{definition}{Definition}

\title{An Introduction to ``Dimensionality''\\Some Warping Required}
\author{Jan Armendariz-Bones}
\date{Last Updated: \today}

\begin{document}
\maketitle
\tableofcontents
\setcounter{section}{-1}
\section{Introduction}
This paper is about the ideas of dimensionality, both in the relatively physical sense of Linear Algebra as well as into the non-standard way of thinking for fractals.
\section{A Brief Summary of Linear Algebra}
First we begin by reviewing some general axioms and definitions from linear algebra.

\begin{definition} (Vector Space Axioms)
		We say that a space $X$ with scalars in a field $\mathbb{F}$ is said to be a vector space if the following conditions are satisfied:
		\begin{itemize}
				\item (Associativity) For any $u,v,$ and $w$ in the space $X$, we have that \[u+(v+w)=(u+v)+w.\]
				\item (Commutativity) For any $u,v\in X$ we have that \[u+v=v+u.\]
				\item (Identity Element) There exists a vector $e\in X$ such that for any $v\in X$, $ev=ve=v$.
				\item (Zero Vector) There exists a vector $0\in X$ such that for any $v\in X$, $0+v=v+0=v$.   
				\item (Inverse Element) For any $v\in X$, there exists another vector -- denoted as the element $-v$ -- such that
						\[
								v+(-v)=0.
						\]
				\item (Associativity of Scalars)For any two scalars $a,b\in\mathbb{F}$ and $v\in X$,
						\[
								a(bv) = b(av).
						\]
		\item (Identity scalar) There exists a $1\in\mathbb{F}$ such that for any $v\in X$, \[1v=v.\] 
		\end{itemize}	
\end{definition}	
\noindent\textbf{Remark:} If we have a subset, $Y$,  of a vector space, $X$,  that satisfies the vector space axioms, we call $Y$ a \emph{subspace} of $X$. 

When we consider any vector space, we need some way to refer to an arbitrary point in space. In $\mathbb{R}^2$, we would say that a point in the space is described as $(x,y)$ where $x$ and $y$ are real numbers, denoting how far they are from the origin with respect to the $x$ and $y$ axes. Now within that description, there was a fundamental choice between the ``building blocks'' of the vector space. Those blocks being $(1,0)$ and $(0,1)$ respectively. How can we determine what we can choose as those blocks? 
\begin{definition}
		(Linear Combination) For any two vectors, $u$ and $v$, we call the linear combination of the two the collection of vectors such that we have vectors of the form \[au+bv,\] where $a$ and $b$ are scalars.
\end{definition}	
\begin{definition}
		(Span) We say that a collection of vectors $\{v_1,v_2,\dots,v_n\}$ span a vector space $V$ when:
		\begin{itemize}
				\item Every vector in space can be represented as a linear combination of our basis vectors. 
		\end{itemize}	
\end{definition}	
\textbf{Remark:} We call the collection of vectors that span the vector space a \emph{basis} for the space $V$. 
\subsection{Dimension of Vector Spaces}
We say that a vector space has a finite dimension if there are a finite number of vectors in the basis. Otherwise we say that the vector space is \emph{infinite dimensional}.

So, if a vector space $\mathcal{V}$  has $3$ vectors in its basis, then we say that $\mathcal{V}$ has a dimension of $3$.  

\section{Manifolds}
\subsection{What is a manifold?}

\textbf{Example to strive towards:} An example of a manifold would be the Earth, as when we ``zoom'' in, the surface of the object appears to look like $\mathbb R^2$.
\subsection{Atlases and Charts}

A \textbf{homeomorphism} is a concept in topology that describes a strong form of equivalence between topological spaces, meaning that they are structurally the same space.
\begin{definition}
		Given two topological spaces \( X \) and \( Y \), we say a function \( f: X \rightarrow Y \) is called a \textbf{homeomorphism} if the following are true:

		\begin{itemize}
				\item Bijective: \( f \) is a one-to-one (injective) and onto (surjective) function.
				\item Continuous: The function \( f \) is continuous, meaning that the preimage of any open set in \( Y \) is open in \( X \).
				\item Continuous inverse: The inverse function \( f^{-1}: Y \to X \) is also continuous.
\end{itemize}

If such a function \( f \) exists, then the spaces \( X \) and \( Y \) are said to be \textbf{homeomorphic}.
\end{definition}	

Said in another way, if these two spaces are homeomorphic, we can treat the spaces like Play-Doh, where we can stretch and warp $X$ into a space that looks like $Y$.  

In topology, an atlas on a manifold \( M \) is a collection of charts that together describe the structure of the manifold.

\begin{definition}
		An atlas for a topological space \( M \) is a collection of pairs \[ \{(U_i, \phi_i)\}_{i \in I}, \] where:

		\begin{itemize}
				\item\( \{U_i\} \) is an open cover of \( M \), meaning that \( M = \bigcup_{i \in I} U_i \) and each \( U_i \) is an open subset of \( M \).
				\item Each \( \phi_i: U_i \to V_i \subset \mathbb{R}^n \) is a \textbf{homeomorphism} from \( U_i \) onto an open subset \( V_i \) of \( \mathbb{R}^n \).
		\end{itemize}
\end{definition}
The maps \( \phi_i \) are known as \emph{coordinate charts}. An atlas thus provides a way of covering the manifold with coordinate systems that describe its local geometry. When the transition maps between overlapping charts \( \phi_i \) and \( \phi_j \) are differentiable (or satisfy other specified conditions), the atlas is said to be differentiable, giving rise to a \emph{differentiable manifold} or \emph{smooth manifold}.

\subsubsection{Open Mapping Condition} % I believe this is the name...
\newpage
\section{Box Covering Dimension}
\subsection{Looking at the Coast of Britain and Fractals}
\section{Hausdorff Covering Dimension}
\subsection{Infinite ``Area''; Zero Area}
%\section{Outline}
%\begin{itemize}
%		\item General Linear Algebra definition of dimensionality.
%				\begin{itemize}
%						\item LADR (uses term degree):
%								
%										A polynomial $p\in P(\mathcal{F})$ is said to have degree $m$ if there exists scalars $(a_i)_1^n$ -- where $a_n\neq \mathbf{0}$ in the scalar field such that for any $z\in\mathcal{F})$, we have that \[p(z)=\sum_{1}^n a_i b_i.\] 
%
%						\item 
%				\end{itemize}	
%		\item Manifolds and atlases.
%				\begin{itemize}
%				\item Abstract Algbera (Visual Algbera): The dimension of a field is the same as the vector space dimension, however the adjoining of an element ``adds'' a dimension to the field.
%						\begin{itemize}
%								\item Consider $\mathbb{R}(i)\cong \mathbb{C}$. From the perspective of the field $\mathbb{R}$, there is two dimensions; with the bases 1 and $i$.
%						\end{itemize}
%				\begin{itemize}
%						\item Hausdorff dimension: How can I separate these two things??
%								\begin{itemize}
%										\item Consider the difference between a (Koch) fractal and a square. One seems to take up less area than the other?
%												\textbf{not a trick question! $A(C)<A(Sq)$}.
%																						\item What is the ``minimum'' hypervolume growth necessary for each?!?
%												\item Why is this fractal so wacky?!
%																														\item Remark that there is infinite perimeter, yet finite area...
%														
%								\end{itemize}	
%				\end{itemize}	
%		\end{itemize}	
%\end{itemize}	
\end{document}
